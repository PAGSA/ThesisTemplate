
%TABLES
\usepackage{dcolumn} %allows alignment of decimal points
\usepackage{longtable} %allows table to be split over pages with headers added automatically
\usepackage{pdflscape}% for making landscape tables
\usepackage{afterpage}% ditto
\usepackage{threeparttable}
\usepackage{multirow}
\usepackage{booktabs}
\usepackage{tabularx}        % Package used to make variable width-columns, i.e.,
							 % column widths are changed to fit the maximum width
							 % and text is wrapped nicely.


%MATH
\usepackage{amsmath}
\usepackage{amsthm}
\usepackage{amssymb}
\usepackage{cases}           % to make numbered cases (equations)
\usepackage{slashed}		 % slash notation
\usepackage{mathrsfs}		 % allows for script letters \mathscr{}

%TEXT
\usepackage{xspace}
\usepackage{textcase}
\usepackage{setspace}        % for use of \singlespacing and \doublespacing
\usepackage{moreverb}        % Using this package to get better control of the
							 % verbatim environment, mostly for the use of the
							 % listing environment which puts line number
							 % beside each line.  Note that there has to be a number
							 % in each set of brackets, i.e., \begin{listing}[1]{1}.
							 % pdf info file is "The moreverb package" by
							 % Robin Fairbairns (rf@cl.cam.ac.uk) after
							 % Angus Duggan, Rainer Schopf and Victor Eijkhout, 2000/06/29.
							 
\usepackage{verbatim}        % allows the use of \begin{comment} and \end{comment}
							 % as well as \verbatiminput{file}
							 % Note:  when using verbatim to input from a text file,
							 % such as a specification or code, use \begin{singlespacing}
							 % and \end{singlespacing}.  Also, tabs are not read
							 % properly, so the input file must only use spaces.

                             %\begin{comment}
                             %Can also use the verbatim package for
                             %comments like this...
                             %\end{comment}

%GRAPHICS
\usepackage{wasysym} %astro symbols
\usepackage{graphics}
\usepackage{graphicx}
\usepackage{boxedminipage}   % to make boxed minipages, i.e., boxes around figures
\usepackage{rotate}          % for use of \begin{sideways} and \end{sideways}
\usepackage{float}           % Using this package to get better control of my floats
% including the ability to define new float types for
% my specification and code listings.
% dvi info file is "An Improved Environment for Floats"
% by Anselm Lingnau, lingnau@tm.informatik.uni=frankfurt.de
% 1995/03/29.


%FOOTNOTES
\interfootnotelinepenalty=10000 % This line stops footnotes from splitting onto two pages.


%MISCELLANEOUS
\usepackage{layout}          % useful for determining the margins of a document
							 % use with \layout command

\usepackage{changebar}       % Way of indicating modifications by putting bars in the
							 % margin.  Read about it in "The Latex Companion".
							 
\usepackage{pdfpages}        %add existing pdf pages

\usepackage{geometry}        %adjust geometry of specific page as need
\usepackage[export]{adjustbox}[2011/08/13]

\usepackage{notoccite}      %Ignores citations in list of figures/tables for numbering purposes

\usepackage[compat=1.0.0]{tikz-feynman} %Custom Feynmann diagrams, see https://www.sharelatex.com/learn/Feynman_diagrams for tutorial
\usepackage{feynmp-auto}
\usepackage{caption}
\usepackage{subcaption}
\newcommand{\degree}{\ensuremath{^\circ}}  %because who wants to type ^circ all the time

% REFERENCES
\usepackage{varioref}           % Better page references, e.g., "on preceding page", etc.
								% \vref{key} Create an enhanced reference.
								% \vpageref[text]{key} Create an enhanced page reference.
								% \vrefrange{key}{key} Create an enhanced range of references.
								% \vpagerefrange[text]{key}{key} Create an enhanced range of page references.
								% Note: doesn't really work for consecutive pages.
% Renewing the text for before and after, because I don't like the default flip-flopping one.
% And 'on the page before' sounds dumb!

\renewcommand{\reftextafter}{on the next page}
\renewcommand{\reftextbefore}{on the previous page}
%-------------------------------------------------------------------------------------------------------------
\usepackage{url}             % for use of \url - pretty web addresses

% HEADINGS
%*********************************************************************************************************
% This changes the headings go that they are prettier, this can be commented out for traditional headings.
\usepackage{sectsty}
\allsectionsfont{\bfseries}% set all the section font to bfseries
\chapterfont{\centering\Large} % set the sizes of chapters, sections ...
\sectionfont{\normalsize}
\subsectionfont{\normalsize}

% for formatting Table of Contents entry, example: Chapter 1 Introduction
\usepackage{tocloft}
\renewcommand{\cftchappresnum}{Chapter }
\renewcommand{\cftchapaftersnum}{:}
\renewcommand{\cftchapnumwidth}{7em}

% for formatting Table of Contents entry for Appendix, example: Appendix 1: Stuff
\newcommand*\updatechaptername{%
	\addtocontents{toc}{\protect\renewcommand*\protect\cftchappresnum{Appendix }}
}

%*************************************************************************************************************
% GLOSSARY
% Using a glossary is more than beginners need to know; leaving the packages, etc. here for now.
%*************************************************************************************************************
%\usepackage[nonumberlist]{glossaries} % use glossaries since glossary package is out dated
%\makeglossaries  % tell latex to make the glossary
%\glossarystyle{list}

%*************************************************************************************************************
%*************************************************************************************************************
% INDEX
% Also possible to make an index; didn't use for my thesis.
%*************************************************************************************************************
%\usepackage{makeidx}         % to make the index
%-------------------------------------------------------------------------------------------------------------
% Tell Latex to make an index
%\makeindex
%*************************************************************************************************************

%\usepackage[sorting=none]{biblatex}
%*************************************************************************************************************
% HYPERLINKS (must be last to work properly, change at own risk)
%*************************************************************************************************************
\usepackage{xcolor}
\usepackage{cleveref}
\usepackage[]{hyperref}
\hypersetup{
	colorlinks,
	linkcolor={blue},
	citecolor={red},
	urlcolor={olive}
}