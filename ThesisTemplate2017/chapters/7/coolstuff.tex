\startchapter{Physics Stuff}
\label{stuff}

I added this chapter to demonstrate some features I added to the physics template.
\section{Feynmann diagrams}
\begin{tikzpicture}
\begin{feynman}
\vertex (a1) {\(u\)};
\vertex[right=5cm of a1] (a2){\(u\)};
\vertex[below=3mm of a1] (a3){\(d\)};
\vertex[right=5cm of a3] (a4){\(d\)};
\vertex[below=3mm of a3] (a5){\(d\)};
\vertex[right=1.5cm of a5] (a6);
\vertex[right=5cm of a5] (a7){\(u\)};
\vertex[below right=of a6] (b1);
\vertex[right=of b1] (c1){\(e^{-}\)};
\vertex[below right=of b1](c2){\(\overline \nu_{e}\)};

\diagram* {
	{[edges=fermion]
		(a1) -- (a2),
		(a3) -- (a4),
		(a5) -- (a6) -- (a7),
		(b1) -- (c1),
		(b1) -- (c2)
	},
	(a6) -- [boson, edge label=\(W^{-}\)] (b1),
};
\end{feynman}
\end{tikzpicture}
\\
See more examples \href{https://www.sharelatex.com/learn/Feynman_diagrams}{here}. Other packages for feynmann diagrams do exist, this is the only one I can get to work with texstudio on windows 10.\\
\section{better hyperlinks}
Set colours at the end of macros/usepackages.tex. I recommend dark colours or black to look more professional. I set the colours bright for now for visibility. \\
urls: \url{google.com}\\
internal links: \ref{stuff}\\
citations: \cite{atkin}\\
Basic colour choices are red, green, blue, cyan, magenta, yellow, black, gray, white, darkgray, lightgray, brown, lime, olive, orange, pink, purple, teal, violet.
Custom colours also possible. Documentation \href{http://muug.ca/mirror/ctan/macros/latex/contrib/xcolor/xcolor.pdf}{here}.

\section{math scripts}
 Addition of slash notation($\backslash$slashed\{\}) and script letters($\backslash$mathscr\{\}):
 \begin{equation}
 \mathscr{L}, \mathscr{H},\slashed{\partial}
 \end{equation}
 \section{tables}
 The book tabs option allows for slightly better looking tables:\\
\begin{table}[h]
	\centering
	\begin{tabular}{ccc}\hline
		Heading 1 & Heading 2 & Heading 3 \\ \hline
		a & b & c \\ 
		d & e & f\\ \hline
	\end{tabular}
	\caption{Regular Table}
\end{table}
\begin{table}[h]
	\centering
	\begin{tabular}{ccc}\toprule
		Heading 1 & Heading 2 & Heading 3 \\ \midrule
		a & b & c \\
		d & e & f\\ \bottomrule
	\end{tabular}
	\caption{Booktabs Table}
\end{table}
\\
Either than spacing the other main difference is a slight bolding of the upper and lower lines. The dcolumn option allows alignment on decimal point:
\begin{table}[h]
	\centering
	\begin{tabular}{|c|c|c|}\hline
		10.2 & 12.24 & 1000.2 \\ 
		54 & 2.8 & 4.35\\ \hline
	\end{tabular}
	\caption{Regular Table}
\end{table}
\begin{table}[h!]
	\centering
	\newcolumntype{d}{D{.}{.}{-1}}
	\begin{tabular}{|d|d|d|}\hline
		10.2 & 12.24 & 1000.2 \\ 
		54 & 2.8 & 4.35\\ \hline
	\end{tabular}
	\caption{dcolumn Table}
\end{table}